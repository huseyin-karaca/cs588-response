\documentclass[11pt]{article}

% ------------------------------------------------
% 1. Packages
% ------------------------------------------------
\usepackage[utf8]{inputenc}
\usepackage[T1]{fontenc}
\usepackage[margin = 1in]{geometry}
\usepackage{xcolor}
\usepackage{amsmath}
\usepackage[most]{tcolorbox}
\usepackage{xr}         % cross-document refs
\usepackage{clipboard}  % read manuscript.cpy
\usepackage{etoolbox}   % \docsvlist
\usepackage{nameref}
\usepackage{hyperref}
\usepackage{fontawesome}

% ------------------------------------------------
% 2. Manuscript Connection
% ------------------------------------------------
\externaldocument{latex_build/main_round2}
\openclipboard{latex_build/manuscript}

% ------------------------------------------------
% 3. Color Palette
% ------------------------------------------------
\definecolor{boxframe}{RGB}{40, 60, 100}
\definecolor{boxfill}{RGB}{250, 250, 252}
\definecolor{revbg}{RGB}{250, 240, 240}
\definecolor{revtitle}{RGB}{180, 50, 50}
\definecolor{authbg}{RGB}{235, 235, 235}
\definecolor{authtitle}{RGB}{0, 80, 120}
\definecolor{revisionbg}{RGB}{240, 248, 255}

% ------------------------------------------------
% 4. Environments & Macros
% ------------------------------------------------
\newtcolorbox{ReviewCycle}[1]{
    enhanced, breakable, width=\textwidth,
    colframe=boxframe, colback=boxfill,
    coltitle=white, fonttitle=\bfseries\sffamily,
    title={#1}, arc=3pt, boxrule=0.8pt,
    top=5pt, bottom=5pt, left=5pt, right=5pt,
    drop fuzzy shadow,
    before upper={\parskip 0.5em}
}

\newcommand{\ReviewerText}[1]{%
    \begin{tcolorbox}[
        enhanced, frame hidden, colback=revbg,
        left=5pt, right=5pt, top=5pt, bottom=5pt, arc=2pt
    ]
        \textbf{\textcolor{revtitle}{}}
        \itshape \textcolor{black!80}{#1}
    \end{tcolorbox}
    \vspace{0.05cm}
}

\newcommand{\AuthorResponse}[1]{%
    \begin{tcolorbox}[
        enhanced, frame hidden, colback=authbg,
        left=5pt, right=5pt, top=5pt, bottom=5pt, arc=2pt
    ]
        \textbf{\textcolor{authtitle}{Response:}}
        \normalfont \textcolor{black}{#1}
    \end{tcolorbox}
    \vspace{0.1cm}
}

\makeatletter
\newcommand{\AutoRevision}[1]{%
    \begin{tcolorbox}[
        enhanced, breakable,
        colback=revisionbg, frame hidden,
        borderline west={3pt}{0pt}{blue!40},
        left=5pt, right=5pt, top=5pt, bottom=5pt,
        arc=2pt
    ]
        \textbf{\textcolor{blue!40!black}{Revisions in Manuscript:}}
        
        \global\def\RevPrintedList{}
        \def\do##1{%
            % Trim spaces
            \begingroup
            \edef\x{\endgroup
                \noexpand\ProcessRevision{##1}%
            }\x
        }%
        \docsvlist{#1}
    \end{tcolorbox}
}

\newcommand{\ProcessRevision}[1]{%
    \edef\RevHash{%
        \ifdefined\pdfmdfivesum
            \pdfmdfivesum{\detokenize\expandafter{\Paste{text:#1}}}%
        \else
            #1%
        \fi
    }%
    \xifinlist{\RevHash}{\RevPrintedList}{%
        % Already printed, skip
    }{%
        \listgadd{\RevPrintedList}{\RevHash}%
        \par\vspace{0.3cm}
        {\small\bfseries\color{blue!60!black} $\hookrightarrow$ Page \pageref{rev:#1}, \nameref{rev:#1}:}
        \par\nopagebreak
        \itshape\small ``\Paste{text:#1}''%
    }%
}
\makeatother

\newcommand{\NoRevision}{%
    \begin{tcolorbox}[
        enhanced, colback=revisionbg, frame hidden,
        borderline west={3pt}{0pt}{gray!50},
        left=5pt, right=5pt, top=5pt, bottom=5pt,
        arc=2pt
    ]
        \textbf{\textcolor{gray!70!black}{Revisions in Manuscript:}} No change was required in the manuscript.
    \end{tcolorbox}
}

% ------------------------------------------------
% 5. Document Content
% ------------------------------------------------

\title{Responses to the  Reviews of the Reviewers}
\begin{document}

\maketitle
\noindent

We thank the Associate Editor and the reviewers for their careful reading of our paper ``Graph Enhanced Transformer for Semi-Supervised Duplicate Bug Report Detection" and their constructive  Reviews. We also thank the reviewers for pointing out our novel contributions. In the revised version of the paper (and in the response document below), we have addressed each of the issues raised by the reviewers. Below, we provide a point by point response to the  Reviews of the reviewers and how they are handled in the revised manuscript. Since we have answered each  Review separately, there are some overlapping answers. Significant changes in the revised paper are also marked in blue.


We thank the editor and reviewers for their thorough and constructive feedbacks. We have carefully addressed all the  Reviews. In the revised document, we have made the following main improvements to the manuscript:


\begin{enumerate}
    \item \textbf{Expanded Future Work and Limitations:} Addressed reviewer suggestions regarding PCA sensitivity, label-scarcity experiments, and hard negative mining by providing detailed discussions in the expanded ``Future Work" and ``Threats to Validity" sections.
    
    \item \textbf{Performance and Reproducibility:} Contextualized the competitive performance of our semi-supervised framework and fully updated the GitHub replication package to ensure complete transparency and reproducibility.
    
    \item \textbf{Quality Control and Formatting:} Conducted a comprehensive manual review of the manuscript to correct grammatical errors, duplicated phrases, and terminological inconsistencies pointed out by the reviewers. Substantially restructured the ``Problem Definition and Motivation" section to eliminate redundancies and include concrete examples of LLM failures and formal DBRD definitions.
\end{enumerate}

We hope these revisions address all reviewer concerns and significantly improve the quality and clarity of the manuscript.


% Buradan sonra Reviewer 1, 2... şeklinde devam edebilirsiniz.
% ==============================================================================
% REVIEWER 1
% ==============================================================================
\clearpage
\section*{Reviewer 1}
\textbf{Overall Recommendation: Weak Accept}
\bigskip

\begin{ReviewCycle}{ Review 1.1}
    \ReviewerText{In Section 5.4 (lines 955--970), the paper discusses the training process by only mentioning starting and ending loss values. It would be more helpful to reference the figures before starting the discussion or include loss curves at the same page to improve reading flow.}
    \AuthorResponse{We thank the reviewer for this valuable suggestion. We agree that explicitly referencing the convergence figures improves both clarity and flow. We have revised the Training Dynamics Analysis section to include explicit references to Figures 3 and 4 at the beginning of the discussion.}
    \AutoRevision{1.1}
\end{ReviewCycle}

\begin{ReviewCycle}{ Review 1.2}
    \ReviewerText{Potential improvements include experimenting with pretrained embeddings or learned node features instead of one-hot encodings, testing different PCA dimensionalities, and conducting ablations for hyperparameters such as projection dimension D, fusion weight $\lambda$, margin m, and training epochs.}
    \AuthorResponse{We thank the reviewer for these thoughtful suggestions. We fully acknowledge the value of these ablation studies and sensitivity analyses. Due to computational resource and time constraints, we were unable to conduct comprehensive hyperparameter exploration in this revision. However, we have now explicitly discussed these directions in the expanded Future Work section.}
    \AutoRevision{1.2}
\end{ReviewCycle}

\begin{ReviewCycle}{ Review 1.3}
    \ReviewerText{Grammatical issues: GCN used before definition, keyword capitalization, ``ie.'' format, duplicated ``for example for example'', missing commas, and ``token id's'' formatting.}
    \AuthorResponse{We thank the reviewer for bringing these issues to our attention. We have systematically corrected all mentioned errors throughout the manuscript.}
    \AutoRevision{1.3a,1.3b,1.3c,1.3d}
\end{ReviewCycle}

\begin{ReviewCycle}{ Review 1.4}
    \ReviewerText{The replication package is missing key files and the training notebook shows interrupted execution.}
    \AuthorResponse{We thank the reviewer for this observation. We have updated our GitHub repository to include all necessary files for complete reproducibility and clarified the repository pointer in the manuscript.}
    \AutoRevision{1.4}
\end{ReviewCycle}

% ==============================================================================
% REVIEWER 2
% ==============================================================================

\clearpage
\section*{Reviewer 2}
\textbf{Overall Recommendation: Weak Reject}
\bigskip


\begin{ReviewCycle}{ Review 2.1}
    \ReviewerText{The paper claims to address ``label-scarce'' environments but uses full datasets. Experiments with reduced training set sizes (5\% or 10\% of labeled data) are needed.}
    \AuthorResponse{We thank the reviewer for this important observation. We acknowledge that our current experiments do not directly validate performance under extreme label scarcity. While the graph-based framework is designed to leverage unlabeled data, we agree that experiments with systematically reduced label budgets would provide stronger evidence. Due to time and computational constraints, we could not complete these experiments for this revision. We have added this explicitly to the Future Work section.}
    \AutoRevision{2.1}
\end{ReviewCycle}

\begin{ReviewCycle}{ Review 2.2}
    \ReviewerText{Table 4 reports only training time overhead. Inference latency metrics must be provided to support scalability claims.}
    \AuthorResponse{We thank the reviewer for this valuable point. We agree that inference latency measurements would strengthen RQ2. We have clarified that the GNN is discarded at inference time and added quantitative inference benchmarking as future work.}
    \AutoRevision{2.2a,2.2b}
\end{ReviewCycle}

\begin{ReviewCycle}{ Review 2.3}
    \ReviewerText{The PCA dimensionality reduction from 768 to d=10 appears arbitrary. Sensitivity analysis should be conducted.}
    \AuthorResponse{We thank the reviewer for raising this concern. We acknowledge that d=10 was empirically motivated but not systematically validated. Due to computational constraints, we have expanded the Future Work section to explicitly highlight the need for PCA dimensionality sensitivity analysis.}
    \AutoRevision{2.3}
\end{ReviewCycle}

\begin{ReviewCycle}{ Review 2.4}
    \ReviewerText{Table 5 shows the method provides no concrete benefit over baselines; results are identical or slightly worse.}
    \AuthorResponse{We thank the reviewer for this observation. We respectfully emphasize that our goal is to demonstrate that a novel graph-enhanced semi-supervised framework can achieve \emph{competitive} performance while explicitly leveraging unlabeled data. Achieving F1 scores that match or closely approximate state-of-the-art models validates the architectural concept. We have added a paragraph emphasizing this perspective.}
    \AutoRevision{2.4}
\end{ReviewCycle}

\begin{ReviewCycle}{ Review 2.5}
    \ReviewerText{While the GNN application shows originality, similar hybrid architectures exist in other NLP domains, limiting foundational novelty.}
    \AuthorResponse{We thank the reviewer for this fair assessment. We acknowledge that hybrid architectures have been explored elsewhere. However, to the best of our knowledge, this work represents an early exploration of graph-based semi-supervised learning specifically for duplicate bug report detection, with distinct domain characteristics and design choices.}
    \NoRevision
\end{ReviewCycle}

\begin{ReviewCycle}{ Review 2.6}
    \ReviewerText{The paper is well-written. However, ``for example'' is duplicated in Line 120.}
    \AuthorResponse{We thank the reviewer for the positive feedback. We have corrected the duplicated phrase.}
    \AutoRevision{2.6}
\end{ReviewCycle}

% ==============================================================================
% REVIEWER 3
% ==============================================================================
\clearpage
\section*{Reviewer 3}
\textbf{Overall Recommendation: Weak Reject}
\bigskip

\begin{ReviewCycle}{ Review 3.1}
    \ReviewerText{Graph edges use only titles. Ignoring descriptions may discard important relationships.}
    \AuthorResponse{We thank the reviewer for this insightful observation. We fully agree that descriptions could yield richer structures. Our choice was motivated by computational efficiency. We have expanded Future Work to discuss incorporating descriptions.}
    \AutoRevision{3.1}
\end{ReviewCycle}

\begin{ReviewCycle}{ Review 3.2}
    \ReviewerText{PCA reduction from 768 to 10 dimensions is very aggressive and may not generalize.}
    \AuthorResponse{We thank the reviewer for this valid concern. We acknowledge the aggressive reduction and its generalization risks. We have expanded Future Work to highlight the need for PCA dimensionality ablation studies.}
    \AutoRevision{3.2}
\end{ReviewCycle}

\begin{ReviewCycle}{ Review 3.3}
    \ReviewerText{Negative sampling may create mostly easy negatives. Hard negative mining could improve decision boundaries.}
    \AuthorResponse{We thank the reviewer for this point. While our approach combines two strategies to ensure balance, we acknowledge that explicit hard negative mining could further sharpen boundaries. We have added this to Future Work.}
    \AutoRevision{3.3}
\end{ReviewCycle}

\begin{ReviewCycle}{ Review 3.4}
    \ReviewerText{Maintaining the full graph in memory may become difficult as repositories grow.}
    \AuthorResponse{We thank the reviewer for this practical concern. We clarify that the GNN is used only during training, not inference. For even larger repositories, we discuss strategies in the expanded Future Work section.}
    \AutoRevision{3.4}
\end{ReviewCycle}

\begin{ReviewCycle}{ Review 3.5}
    \ReviewerText{Evaluation limited to Eclipse and Thunderbird constrains generalization claims.}
    \AuthorResponse{We thank the reviewer for this valid observation. We acknowledge this limitation and have listed evaluation on diverse datasets as future work.}
    \AutoRevision{3.5}
\end{ReviewCycle}

\begin{ReviewCycle}{ Review 3.6}
    \ReviewerText{Overlap between Introduction, Problem Description, and Related Work sections.}
    \AuthorResponse{We thank the reviewer for this structural feedback. We have substantially revised Section 2 to focus on problem definition and motivation, reducing redundancy.}
    \AutoRevision{3.6}
\end{ReviewCycle}

\begin{ReviewCycle}{ Review 3.7}
    \ReviewerText{Typos and awkward phrases, such as ``for example for example''.}
    \AuthorResponse{We thank the reviewer. We have corrected the duplicated phrase and reviewed the manuscript for similar issues.}
    \AutoRevision{3.7}
\end{ReviewCycle}

\begin{ReviewCycle}{ Review 3.8}
    \ReviewerText{Results don't beat baselines; lack of cross-validation or statistical tests makes small differences unreliable.}
    \AuthorResponse{We thank the reviewer. We acknowledge both concerns: (1) our goal is competitive performance with unlabeled data leverage, and (2) lack of statistical rigor is a limitation. We have acknowledged this in Threats to Validity.}
    \AutoRevision{3.8}
\end{ReviewCycle}

% ==============================================================================
% REVIEWER 4
% ==============================================================================
\clearpage
\section*{Reviewer 4}
\textbf{Overall Recommendation: Weak Accept}
\bigskip

\begin{ReviewCycle}{ Review 4.1}
    \ReviewerText{Section 2 reads like methodology rather than problem definition and motivation.}
    \AuthorResponse{We thank the reviewer for this structural critique. We have substantially revised Section 2 to focus on problem definition and technical motivation.}
    \AutoRevision{3.1}
\end{ReviewCycle}

\begin{ReviewCycle}{ Review 4.2}
    \ReviewerText{Graph construction ignores descriptions, which may forgo significant data.}
    \AuthorResponse{We thank the reviewer. We acknowledge this limitation and have listed incorporating descriptions as future work.}
    \AutoRevision{3.3}
\end{ReviewCycle}

\begin{ReviewCycle}{ Reviews 4.3--4.10}
    \ReviewerText{Various formatting and terminological issues throughout the manuscript.}
    \AuthorResponse{We thank the reviewer for meticulous attention to detail. We have systematically corrected all mentioned issues.}
    \AutoRevision{4.5,4.6,3.6}
\end{ReviewCycle}

\begin{ReviewCycle}{ Review 4.11}
    \ReviewerText{Section 2 should include concrete examples of LLM failures and properly describe the DBRD problem.}
    \AuthorResponse{We thank the reviewer. We have substantially revised Section 2 to include formal problem definition and concrete motivation with realistic scenarios.}
    \AutoRevision{3.1,4.2}
\end{ReviewCycle}

\begin{ReviewCycle}{ Reviews 4.12--4.28}
    \ReviewerText{Various additional issues: abbreviations, phrasing, figure references, early stopping details, and scalability discussions.}
    \AuthorResponse{We thank the reviewer for the comprehensive list. We have systematically addressed all issues including clarifying early stopping criterion, adding figure references, and discussing distributed strategies.}
    \AutoRevision{1.7,4.9,1.5,4.12,4.13,3.5}
\end{ReviewCycle}



\end{document}
